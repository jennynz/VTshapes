\section{Intra-speaker PCA correlations}

In order to quantify individual speaker effects, the same PCA and correlation analysis was carried out between vowel sets from the same speaker, when there were two sets of repeated raw MRI data available. We had access to two sets of raw MRI data for each speaker, with a few exceptions where Set 2 was incomplete (e.g. missing a vowel, only having repeats of the cardinal vowels, having less than 13 sagittal slices). Former Masters student Daniel Tan had already processed both Set 1 and Set 2 for five speakers (VT03, VT05, VT08, VT09, VT10) \cite{daniel}. Given the time-consuming nature of the manual steps in extracting these area functions, and the gaps in the data set where full second sets were not available, it was decided that second MRI image sets would only be processed for VT11, VT04 and VT07 to add to the intra-speaker correlation analysis. These three speakers were selected as their Set 1 area functions showed consistently high inter-speaker correlations.

\subsection{Artifacts in VT11}

Speaker VT11's MRI images included artifacts in the oral cavity around the front of the tongue due to a metallic implant in their mouth. Since VT11's two sets were analysed by Helen Searle \cite{helen} and myself, some exploratory tests were performed such as isolating the pharyngeal region to see if intra-speaker correlations would improve when the affected area is excluded. This hypothesis assumed that since the artifacts which disrupted the cross-sectional slices were in the oral cavity, Helen and I may have had different strategies for getting around this during the data point placement step, causing variation between the area functions of the two sets.