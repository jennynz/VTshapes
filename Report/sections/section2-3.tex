\section{Inter-speaker PCA correlations and demographic trends}

[SECTION INCOMPLETE]

The next step was to compare the vowel quadrilaterals generated from the vocal tracts of participants in different age categories, and with different accents. For the latter, American vowel data from Story (2015 or whatever) was used. PC1, PC2 and PC3 were compared across the speakers. Pearson product-moment correlation used. https://statistics.laerd.com/statistical-guides/pearson-correlation-coefficient-statistical-guide.php

In order to analyse trends, the correlation values between different age, gender and accent groups were highlighted and [quantified in some way as being significantly higher/lower within the same demographic than outside of it?]


Since the principal components of two speakers' datasets are entirely independent, this statistical measure was selected to give a measure of their similarity.