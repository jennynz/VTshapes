\section{Area functions of VT04 and VT07}\label{sec:results-areafunctions}

Figures \ref{fig:vt04area} and \ref{fig:vt07area} show the two sets of area functions which were added to the existing data set, from VT04 and VT07 respectively. Area functions for the 11 vowels analysed for VT04 and VT07 are shown, where the x-axis indicates vocal tract length (i.e. the distance from the lips) and the y-axis shows the cross-sectional area, both in millimetres. The vertical blue line indicates the boundary between the oral and pharyngeal cavities. The corresponding mid-saggital MRI image from which the area function was derived is shown to the right of each area function. These pairs are placed in the positions that the vowels would normally be situated on a vowel quadrilateral (Figure \ref{fig:vowelquad}).

The area functions for VT04 and VT07 appeared to be smaller in amplitudes than the other area functions produced by the other vocal tracts (Figure \ref{fig:interp-had}) \cite{helen}. This was somewhat expected, since it was noted by Dr Catherine Watson as she was supervising the MRI acquisition that these speakers did not enunciate or open their mouths as clearly as the other speakers - a reason why they had been left until last to analyse.

Nevertheless, their shapes follow the general trends for each vowel. For example, the front high vowel `heed' showed larger cross-sectional areas in the pharyngeal cavity than the oral cavity, since the tongue would be pushed up near the roof towards the front of the mouth. On the other hand, the back vowels like `hoard' and `hod' have larger oral cavities as the tongue arches towards the back of the mouth.

\begin{landscape}

\begin{figure}[b]
\centering
\hspace*{-1.5cm}
\includegraphics[width=1.2\paperwidth]{img/vt04areaplot}
\caption{Area functions of Vocal Tract 4}
\label{fig:vt04area}
\end{figure}

\pagebreak

\begin{figure}[b]
\centering
\hspace*{-1.5cm}
\includegraphics[width=1.2\paperwidth]{img/vt07areaplot}
\caption{Area functions of Vocal Tract 7}
\label{fig:vt07area}
\end{figure}

\end{landscape}

